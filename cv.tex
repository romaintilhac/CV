% Based on autoCV by Jitin Nair
% https://www.overleaf.com/latex/templates/autocv/scfvqfpxncwb

% % fontspec allows you to use TTF/OTF fonts directly
% \usepackage{fontspec}
% \defaultfontfeatures{Ligatures=TeX}

% % modified for ShareLaTeX use
% \setmainfont[
% SmallCapsFont = Fontin-SmallCaps.otf,
% BoldFont = Fontin-Bold.otf,
% ItalicFont = Fontin-Italic.otf
% ]
% {Fontin.otf}
\documentclass[a4paper,11pt]{article}
\usepackage[francais]{babel}
\AtBeginDocument{\def\labelitemi{$\bullet$}}
\usepackage[T1]{fontenc}
\usepackage{lmodern}
\usepackage{url}
\usepackage{parskip} 	

\RequirePackage{graphicx}
\usepackage[usenames,dvipsnames]{xcolor}
\usepackage[scale=0.9]{geometry}

\usepackage[style=authoryear,sorting=ynt, maxbibnames=2]{biblatex}
\usepackage{supertabular} %to prevent spillover of tabular into next pages
\usepackage{tabularx} %tabularx environment
\usepackage{enumitem}
\usepackage{dirtytalk} %quotation marks
\usepackage{ulem}
\usepackage{titlesec}				
\usepackage{multicol}
\usepackage{multirow}
\usepackage[unicode, draft=false]{hyperref}
\usepackage{color} %colors
    \definecolor{custom}{rgb}{0,0.2,0.6}
    \definecolor{back}{rgb}{0.85, 0.89, 0.95}
    \definecolor{fore}{rgb}{0.56, 0.67, 0.86}
\usepackage{hyperref} %hyperlinks
   \hypersetup{colorlinks=true,allcolors=custom}
\usepackage{fontawesome5} %social icons

\newcolumntype{C}{>{\centering\arraybackslash}X} % centered version of 'X' col. type

\newlength{\fullcollw}
\setlength{\fullcollw}{0.47\textwidth}

\titleformat{\section}{\Large\scshape\raggedright}{}{0em}{}[\titlerule]
\titlespacing{\section}{0pt}{10pt}{10pt}

\setlength\bibitemsep{1em}

\begin{document}

\pagestyle{empty} 

% HEADER
\begin{tabularx}{\linewidth}{@{} C @{}}
\LARGE{\textsc{Romain Tilhac}} \\
Géochimiste $|$ Chercheur post-doctorant (IACT/CSIC, Grenade)\\[7.5pt]
\href{mailto:romain.tilhac@csic.es}{\raisebox{-0.05\height}\faEnvelope \ romain.tilhac@csic.es} \ $|$ \
\href{https://orcid.org/0000-0001-5132-6228}{\raisebox{-0.05\height}\faOrcid \ 0000-0001-5132-6228} \ $|$ \
\href{https://romaintilhac.github.io}{\raisebox{-0.05\height}\faGlobe \ romaintilhac.github.io} \ $|$ \ 
\href{https://github.com/romaintilhac}{\raisebox{-0.05\height}\faGithub\ romaintilhac}
\end{tabularx}

% RESEARCH INTERESTS
\section{Thèmes de recherche}

    {Mes recherche portent sur la genèse et la migration des magmas et les interactions magma-roche et leurs rôles de dans l'évolution et la dynamique du manteau terrestre. Mon approche combine une large gamme de méthodes analytiques et de modèles numériques visant à développer une vision pétrologiquement cohérente de textit{géochimie computationelle}. Je travaille notamment sur la formation et le recyclage des pyroxénites en tant qu'hétérogénéités dans le manteau convectif et leur impact sur la genèse des basaltes océaniques et les grands cycles géochimiques.}
    
% EDUCATION
\section{Cursus universitaire}

    {\bf Doctorat en Pétrologie et Géochimie}, Macquarie University
    \hfill {2013 - 2017}\\
    {\footnotesize Co-tutelle avec l'Université Paul Sabatier (Toulouse)}
    \hfill \textit{Sydney, Australie}\\
    \say{\textit{Petrology and geochemistry of pyroxenites from the Cabo Ortegal Complex, Spain}}
    
    {\bf Licence \& Master en Sciences de la Terre}, Université Paul Sabatier
    \hfill {2006 - 2011}\\
    \uline{Major de promotion, Mention Bien}
    \hfill \textit{Toulouse, France}
     
% EXPERIENCE
\section{Expériences professionnelles}
    
    \textbf{Chercheur post-doctorant \textit{JdC Fellow}}
    \hfill {Depuis 2020}\\
    Instituto Andaluz de Ciencias de la Tierra (IACT)/CSIC, avec \textbf{C. Garrido}
    \hfill \textit{Grenade, Espagne}
     
    \textbf{Chercheur post-doctorant \textit{JSPS Fellow}}
    \hfill {2020}\\
    Kanazawa University, avec \textbf{T. Morishita}
    \hfill \textit{Kanazawa, Japon}
    
    \textbf{\textit{Research Associate}}
    \hfill {2017 - 2019}\\
    ARC Centre of Excellence CCFS/GEMOC, avec \textbf{S.Y. O'Reilly}
    \hfill \textit{Sydney, Australie}\\
    Responsable de l’équipe de géochronologie \textit{TerraneChron}
    
% SKILLS
\section{Compétences scientifiques}

    \textbf{Techniques analytiques} 
    \begin{itemize}[itemsep=0pt,parsep=2pt]
        \item Pétrographie magmatique et minéralogie (microscopie optique, MEB, thermo-barométrie, micro-thermométrie)
        \item Séparation minérale (désagrégation Selfrag, magnétique, liqueurs denses, piquage)
        \item Chimie par voie humide (digestion acide et en tubes Carius, chromatographie par colonne, extraction Re-Os par solvant et micro-distillation, dilution isotopique)
        \item Géochimie élémentaire (analyse en solution/\textit{in situ} et cartographie des éléments majeurs \& traces par EPMA, [LA]-ICP-MS)
        \item Géochimie isotopique et géochronologie (analyse des isotopes radiogéniques Rb-Sr, Sm-Nd, Lu-Hf \& Re-Os par TIMS (Triton) et MC-ICP-MS (Nu Plasma, Neptune), U-Pb/Lu-Hf sur zircon par LA-[MC]-ICP-MS)
    \end{itemize}
    
    \textbf{Modélisation numérique et traitement de données}
    \begin{itemize}[itemsep=0pt,parsep=2pt]
        \item Modélisation de partitionnement élémentaire et fractionnement isotopique par les processus magmatiques
        \item Développement de modèles de diffusion, percolation-diffusion, fusion de sources mixtes \& en système ouvert
        \item Modèles thermo-mécaniques de transport réactif, modélisation thermodynamique (pMELTS, PerpleX, Melt-PX)
        \item Langages de programmation: Matlab, Python, VBA, Julia, Fortran, HTML
    \end{itemize}
    
    \textbf{Géologie de terrain}
    \begin{itemize}[itemsep=0pt,parsep=2pt]
        \item Pétrologie magmatique et métamorphique en terrain mafique et ultramafique
        \item Expériences de terrain : Pyrénées, Galice, sud de l'Espagne, Italie, République Tchèque, Californie, Australie, Terre-Neuve
        \item Analyse microstructurale, échantillonnage et cartographie
    \end{itemize}

% GRANTS & REWARDS
\section{Financements \& récompenses}

    \textbf{Crédits de recherche en tant que PI \textit{\say{Proyectos de generación de conocimiento}}}
    \hfill {Sept. 2022}\\
    \say{\textit{M\underline{o}delling ar\underline{c} r\underline{e}cycling in the oceanic m\underline{a}ntle using radioge\underline{n}ic i\underline{s}otope systems}} (OCEANS)\\
    Ministère espagnol des Sciences, de l'Innovation et des Universités 45 k€ (2 ans)
    
    \textbf{Financement post-doctoral \textit{Juan de la Cierva (Incorporación)}}
    \hfill {Août 2021}\\
    Ministère espagnol des Sciences, de l'Innovation et des Universités (3 ans)
    
    \textbf{Financement post-doctoral \textit{Juan de la Cierva (Formación)}}
    \hfill {Déc. 2019}\\
    Ministère espagnol des Sciences, de l'Innovation et des Universités (2 ans)
    
    \textbf{Financement post-doctoral JSPS (Short-term)}
    \hfill {Oct. 2019}\\
    \textit{Japan Society for the Promotion of Science} (1 an)
    
    \textbf{Thèse de doctorat classée dans les 10\% des meilleures thèses examinées}
    \hfill {Sept. 2017}\\
    Macquarie University (Sydney)
    
    \textbf{Bourse doctorale d’excellence iMQRES}
    \hfill {Févr. 2012}\\
    International Macquarie Research Excellence Scholarship (3,5 ans)

% OTHER ACTIVITIES
\section{Activité scientifique communautaire}

    \textbf{Séminaires \& présentations invités}
    \begin{itemize}[label={},itemsep=0pt,parsep=0pt]
        \item Université Goethe de Francfort (\textit{Geosciences colloquium series})
            \hfill \textit{Francfort}, {Janv. 2023}
        \item CNRS Forsterite workshop 2021 (Modélisation des transferts manteau-croûte)
            \hfill \textit{Pyrénées}, {Oct. 2021}
        \item International Symposium DEEP 2021
            \hfill \textit{Nanjing}, {Oct. 2021}
        \item Université de Tokyo
            \hfill \textit{Tokyo}, {Mar. 2019}
        \item Geoanalysis 2018 workshop (Application du LA-[MC]-ICP-MS à l'exploration)
            \hfill \textit{Sydney}, {Juill. 2018}
    \end{itemize}

    \textbf{Organisation de sessions à la Goldschmidt Conference}\\
    \say{\textit{Insights on the formation, preservation and transport of mantle compositional heterogeneities}} \hfill {2023}\\
    \say{\textit{Mantle heterogeneity: origins and contribution to magmatism and implications for mantle dynamics}} \hfill {2021} \\
    \say{\textit{Development and recycling of chemical and isotopic heterogeneities in the sub-arc mantle}} \hfill {2020}
    
    \textbf{Reviewer fréquent pour des revues scientifiques internationales (24 reviews)}\\
    Geology, J. of Petrology, Earth-Science Reviews, Chemical Geology, Scientific Reports, GSL Special Publications, Lithos, European J. of Mineralogy, EMU Notes in Mineralogy, American J. of Science, Frontiers, Ofioliti

% SUPERVISION & TEACHING
\section{Encadrement \& enseignement}
    \textbf{Thèse d'H. Henry}, Macquarie University
    \hfill {2015 - 2018}\\ 
    \say{\textit{Mantle pyroxenites: deformation and seismic properties}}
    \hfill \textit{Sydney, Australie}

    \textbf{Mémoire de M. Smith}, Macquarie University
    \hfill {2018}\\ 
    \say{\textit{Dating the Donkerhuk granite, Damara Orogen, Namibia}}
    \hfill \textit{Sydney, Australie}

\textbf{Enseignement}, Macquarie University
\hfill {2014-2019}
    \begin{itemize}[label={},itemsep=0pt,parsep=0pt]
        \item Niveau Master : modélisation du comportement des éléments traces durant les processus magmatiques 
        \item Niveau Licence : terrain de géologie structurale et métamorphique (Hill End, Nouvelles-Galles du Sud)
    \end{itemize}

% ADDITIONAL TRAINING
\section{Formations complémentaires}

    \begin{itemize}[label={},itemsep=0pt,parsep=0pt]
        \item Cartographie LA-ICP-MS : applications en pétrologie \& volcanologie (M. Petrelli, C. Stremtan, M. Šala)
        \item Fugacité d’oxygène : théorie et pratiques en géosciences (C.A. McCammon, H.St.C. O’Neill, D.J. Frost)
        \item Analyses et techniques géochimiques (N.J. Pearson)
        \item Frontières de recherche en géophysique et géodynamique (C.J. O’Neil)
    \end{itemize}
    
\vfill  
\center{\footnotesize Mis à jour: \today}

\end{document}
